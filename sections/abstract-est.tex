\begin{titlepage}
\begin{center}
    

{\large \textbf{Luria vaheldutavate seeriate tõlgendatavate anomaaliate ja kinemaatiliste parameetrite analüüs Parkinsoni tõbi  modelleerimisel}}

\vspace*{1cm}

{\Large \textbf{Annotatsioon}}

\vspace*{1cm}
        
\end{center}

Käesoleva töö põhieesmärk oli analüüsida Luria vahelduvate seeriate testide käigus joonistatud mustrid. Töö käigus arvutatakse tunnused mille alusel trennitakse masinõpe mudelid Parkinsoni tõbi diagnoosimiseks.

Psühholoogias ja neuroloogias Luria vahelduvate seeria testid kaustatakse patsiendi motoorika seisu hindamiseks. Testide tulemuste alusel uuritakse haiguse mõju liigutuse planeerimis- ja teostamis-funktsioonidele . Luria vahelduvate seeriate testid kuuluvad biomarkerite hulka, mida kasutatakse  Parkinosni tõbi diagnoosimisel.

Töö põhipanuseks on uus meetod mustrite analüüsimiseks, joonistatud Luuria peenmotoorsete testide käigus. Pakutud meetodi iseloomustab joonistatud mustrite kirjeldamise viis.  Nimelt, meetod võimaldab analüüsida joonistatud mustri erinevate detailiseerimise tasemetel. Selle saavutamiseks  loodi unikaalne masinõpe algoritmite rakendamisjada. Esimeseks sammuks rakendatakse arvuti nägemise algoritm joonistatud mustri põhilelementide (loogiliste segmentide) tuvastamiseks. Selle alusel moodustatakse tunnuste jadad. Jadad kirjeldavad testide tulemused kinemaatiliste parameetrite keeles. Järgmiseks sammuks on tehis närvivõrgu treenimine mille abiga konstrueeritakse oodatud kinemaatiline portree mille alusel leitakse joonistamisanomaaliad.

Põhitulemuseks on klassifikaator mis võimaldab eristada patsiente Parkinsoni tõbiga kontroll rühmast täpsusega 91\%. Samuti pakutud meetod annab võimaluse jälgida otsuse tegemist.   

Töö on kirjutatud Inglise keeles ning sisaldab teksti \pageref{LastPage} leheküljel, 11 peatükki, 16 tabelit ja 19 diagrammi.

\end{titlepage}
